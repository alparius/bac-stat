\section{Introduction}

\subsection{The Romanian Baccalaureate}

The Romanian Baccalaureate \cite{bac} is the exam taken by students in Romania when they graduate high school. For the most part, the baccalaureate consists of three written exams. Each written exam is graded on a scale of 1 to 10 with 10 being the best possible grade. For students to pass the exam as a whole they need to obtain a minimum of 5 on each written module and a final average of at least 6.

Romanian language and literature is one of the three written modules. This exam has to be taken by every exam participant. Moreover, each student must take a mandatory and an elective exam based on their class's study profile. In addition to these three exams, many students who belong to an ethnic minority group are also studying their native language and literature, thus must take a fourth exam in this subject.

\subsection{Minorities of Romania}
\label{ssec:min}

Romania is home to several ethnolinguistic groups who can study their native language and literature in school. The most prevalent minority group in Romania are the Hungarians with 6.3\% of the population, according to the country's last census in 2011 \cite{census}. Other linguistic minorities include the Ukrainian (0.24\%), German (0.13\%) and Turkish (0.13\%) groups.

% For the purposes of this paper, those studying one of the minority languages are treated as a single group because they all have four exams instead of three. In addition, as the children studying in these classes generally have a native language that is not Romanian, they commonly have worse knowledge of Romanian.


\subsection{Motivation}
\label{ssec:motiv}

There are several assumptions that people in Romania hold about the results of the minority groups on the baccalaureate. Some of these suppositions are the following:

\begin{itemize}
\setlength\itemsep{0em}
  \item The pupils belonging to minority groups receive worse final grades and in particular, worse grades in Romanian language and literature. The general argument for the first part of this presumption is that since the minorities need to take an additional exam, they have less time to prepare. On the other hand, people claim that as Romanian is not the native language of the minority groups, they likely have lower grades in this subject.
  \item Better minority students are less affected by their lack of Romanian linguistic knowledge on the Romanian literature exam.
  \item Students who are part of a minority have a worse chance of successfully increasing their Romanian literature grade through the grade appeal process as they get penalised for their grammatical errors that are specific to non-native Romanian speakers.
\end{itemize}

It is worth bearing in mind that these suppositions merely rely on anecdotes and personal observations. Our goal is to challenge these assumptions by utilising the available data.

%Being minority members, most of the assumptions we encountered revolved around comparisons between the main Romanian population and the different ethnolinguistic groups. Having to take a fourth exam while still getting the same full subject on the Romanian one is a recurring pain point, but the other side often argues that this mother tongue language and literature exam is too easy compared to the rest of the examination, raising the averages.

%Another widespread presupposition is that when minorities' appealed Romanian language and literature exams get sent to predominantly Romanian counties (which they very likely are), it probably gets downgraded if the person evaluating the exam recognises the pupil as a minority from their writing style.


% A list of common assumptions:
% \begin{itemize}
% \item The minorities have worse final grades because they have to study for 4 subjects instead of just 3. \emph{(Final grade distribution minority (maybe even Hungarian, German separately) vs Romanian. Final grade minority vs Romanian.}
% \item The minorities have worse scores in Romanian than the Romanians but the minorities’ native language grade raises their final grade. \emph{(PDF of minority vs Romanian grade. Romanian grade plotted against Hungarian grade or Romanian against final grade.)}
% \item The minorities have better foreign language skills.
% \item For better students, being part of the minority makes less of a difference: the Romanian grade is still high - but for worse students it makes it even worse.
% \item Students studying in science classes have better grades than those studying in humanity classes. (Only if there is space)
% \end{itemize}

